%---------------------------------------------------------------------
\subsubsection{Minuta de reuni�o (29-abr-2014)}

\begin{tabbing}
  Local \= xxx \kill
  Local \> : LEAD \\
  Data  \> : 29 de Abril de 2014 \\
  Hora  \> : 14:00
\end{tabbing}

%---------------------------------------------------------------------
\participantes{
  \alana,
  \jacoud,
  \andre,
  \elael,
  \gabriel,
  \julia,
  \patrick,
  \ramon,
  \renan.
}

\pauta{Acompanhamento das atividades.}

\begin{itemize}
  \item Abertura. A reuni�o do Projeto ROSA foi convocada por \ramon.

  \item Aprova��o da minuta da reuni�o anterior. \\

  \item \ramon:
  \begin{itemize}
    \item Pediu um diagrama mostrando a conex�o de todas as �reas do projeto para ser anexado ao Relat�rio de Junho (08/06/2014). A ideia � documentar todas as estruturas que ser�o implementadas no rob� ROSA.
    \item Diagrama: \\
      \begin{itemize}
        \item Desenho mec�nico da viga com todos os componentes.
        \item Eletr�nica: PC embarcado (PC104) ou placa com microcontrolador.
        \item Software: O que ser� usado e como se conectam (sensor por sensor, canal de comunica��o, drives necessaries, conectores, etc.)
      \end{itemize}
      \item Relat�rio dia 5 de Maio:
      \begin{itemize}
        \item Adicionar esbo�o do diagrama t�cnico.
        \item J� foi pedido ao Ant�nio o RAP e o resumo/extrato dos gastos do projeto at� agora.
      \end{itemize}
  \end{itemize}

  \item \elael:
  \begin{itemize}
    \item Implementa��o da interface em andamento (prioridade).
    \item Sensor Indutivo: pouco tempo para finalizar o componente. Enviou email para o Sylvain mas o componente ainda n�o est� completo. Esperar relat�rio para saber se realmente vai acontecer.
  \end{itemize}

  \item \gabriel:
  \begin{itemize}
    \item Contornou o bug do sistema utilizando pontos ao inv�s de cubos, conseguindo visualizar mapas grandes no Octomap. Resolvendo o problema de mem�ria.
    \item Tarefa: documentar a melhor forma de gerar dados antes de tomar uma decis�o definitiva.
  \end{itemize}

  \item \andre:
  \begin{itemize}
    \item Realizou experimentos em bancada. Observou o descarregamento da bateria medindo a tens�o no setup escolhido. Vai entregar um estudo a respeito ao Jacoud.
  \end{itemize}

  \item \renan:
  \begin{itemize}
    \item Fez pesquisa sobre PC104: optou pela ADL (900USD) com todos os cabos e componentes necess�rios.
    \item Fez pesquisa sobre placa CAN (junto com PC 104). Encontrou na Grid Connect. %, 100 dolares de importa��o
    \item Falta fazer revis�o da placa microcontrolada com Jacoud.
  \end{itemize}

  \item \julia:
  \begin{itemize}
    \item Integra��o do aplicativo em andamento 
    \item Documenta��o de teste ok.
  \end{itemize}

  \item \alana:
  \begin{itemize}
    \item Checar status das importa��es de materiais permanentes: PanTilt, Sonar, Eletr�nica Embarcada, PC104 com placa CAN.
    \item Pend�ncia devido � auditoria interna na COPPETEC. Setor volta a operar no dia 5/05.
  \end{itemize}


%  \item Pauta para a pr�xima reuni�o: N�o definida.

\end{itemize}

\vspace{10mm}%
\parbox[t]{70mm}{
  Aprovado por: \\[5mm]
  \centering
  \includegraphics[width=65mm]{../assinatura/assinatura-digital.jpg} \\[-4mm]
  \rule[2mm]{70mm}{0.1mm} \\
  \ramon \\[1mm]
  Coordenador do Projeto \\
}

%---------------------------------------------------------------------
\fim
