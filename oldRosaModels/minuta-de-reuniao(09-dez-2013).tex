%---------------------------------------------------------------------
\subsubsection{Minuta de reuni�o (09-dez-2013)}

\begin{tabbing}
  Local \= xxx \kill
  Local \> : LEAD \\
  Data  \> : 09 de Dezembro de 2013 \\
  Hora  \> : 10:00
\end{tabbing}

%---------------------------------------------------------------------
\participantes{
  \jacoud,
  \andre,
  \elael,
  \gabriel,
  \julia,
  \ramon,
  \renan.
}

\pauta{Acompanhamento das atividades.}

\begin{itemize}
  \item Abertura. A reuni�o do Projeto ROSA foi convocada por \ramon.

  \item Aprova��o da minuta da reuni�o anterior.

  \item Em aberto:
  \begin{itemize}
    \item Criar of�cios para obter as rubricas (Ant�nio e Gizele).
    \item Seguro de Sa�de.
    \item Substituto para Rafael, encontrar candidato.
    \item Reuni�o para discutir conceito. Agendar com os grupos para o dia 16/12. Trazer anota��es. \\
  \end{itemize}

  \item Tarefas para o coordenador.
  \begin{itemize}
    \item Requerimentos para aquisi��o: laptops LENOVO, sensor, sonar e encoder, nacionais e importados.
    \item Of�cios para obter notas de projeto. \\
  \end{itemize}

  \item Tarefas para Jacoud.
  \begin{itemize}
    \item Software de Scrum: ser� o respons�vel por coordenar nosso scrum e quadro de tarefas.
    \item Viabilizar ScrumDo para todo LEAD afim de que tenhamos controle de cada participante em todos os projetos vigentes.
    \item Criar quadro na nossa sala para visualizar tarefas do Projeto ROSA. \\
  \end{itemize}

  \item Sylvain.
  \begin{itemize}
    \item N�o participar� da reuni�o �s 10:00. Ele far� uma reuni�o posterior com o grupo de software durante a tarde. \\
  \end{itemize}

  \item Grupo de design (Julia)
  \begin{itemize}
    \item Entregar requerimento de compra de sonar (ESBR), sensor e codificador.
    \item Coordenar com Ant�nio e Gizele a quest�o das rubricas.
    \item COPPETEC: Seguro de vida da equipe.
    \item Contrato de transfer�ncia para Alemanha do Renan finalizado. \\
  \end{itemize}

  \item Time Software
  \begin{itemize}
    \item \textbf{\gabriel.} Estudou Octoviz para entender funcionamento. Construiu a base do plugin 3D e esta em fase de teste. \\

    \item \textbf{\elael.} Acabou de transcrever o c�digo com a ressalva de que o pixel buffer n�o est� funcionando da mesma forma que funcionava no Windows (pode ser sintaxe ou placa de v�deo). Fez os testes de precis�o do z-buffer e encontrou um problema: funcionamento normal mas a partir de 25 unidades de dist�ncia ele d� o mesmo resultado. Seguindo recomenda��es do Sylvain, vai postar perguntas no forum online. \\
  \end{itemize}

  \item Grupo de pot�ncia
  \begin{itemize}
    \item \textbf{\andre.} Levantou pontos necess�rios para usar o cabo de fibra �tica. J� mandou para o Patrick. Preparar abstract para reuni�o sobre conceito. \\

    \item \textbf{\renan.} Pesquisa sobre sonares e sobre Pan \& Tilt. Conversou com um representante da Kongsberg e ele deu a possibilidade deles gerarem um solu��o espec�fica para o nosso caso. A ser combinado na reuni�o de conceito. \\
  \end{itemize}

%  \item Pauta para a pr�xima reuni�o: N�o definida.

\end{itemize}

\vspace{10mm}%
\parbox[t]{70mm}{
  Aprovado por: \\[5mm]
  \centering
  %\includegraphics[bb=1 1 1238 299,width=65mm]{../assinatura/assinatura-digital.jpg} \\[-4mm]
  \includegraphics[width=65mm]{../assinatura/assinatura-digital.jpg} \\[-4mm]
  \rule[2mm]{70mm}{0.1mm} \\
  \ramon \\[1mm]
  Coordenador do Projeto \\
}

%---------------------------------------------------------------------
\fim
