%-----------------------------------------------------------------------
%
%   UFRJ  - Universidade Federal do Rio de Janeiro
%   COPPE - Coordena��o dos Programas de P�s-gradua��o em Engenharia
%   PEE   - Programa de Engenharia El�trica
%
%
%   Projeto ROSA - Rob� para opera��o de stoplogs alagados
%
%   Minutas de reuni�es
%
%                                                        20/mar/14, Rio
%                                                        Ramon R. Costa
%----------------------------------------------------------------------
\documentclass[12pt,a4paper]{article}
\usepackage{amsmath,amssymb}  %pacotes do AMS
\usepackage[utf8]{inputenc} %pacote para entender palavras acentuadas
\usepackage{latexsym}         %pacote para incluir simbolos (ex.\Box)
\usepackage{fancybox,fancyhdr}%pacote com frescuras
\usepackage{graphicx}         %pacote para incluir figuras tipo eps
\usepackage{xcolor}
\usepackage{xspace}
\usepackage{pst-all,pst-poly}  %PSTricks
\usepackage{psfrag} 
\usepackage{calc}
\usepackage{multicol}
\usepackage[english]{babel} 
\usepackage[a4paper]{hyperref}

 
\input{commands}
\input{a4size} 
\input{macros-ROSA}
  
\begin{document}
%---------------------------------------------------------------------
\pagestyle{fancy}%
\thispagestyle{fancy}%
\renewcommand{\headrulewidth}  {0.4pt}%
\renewcommand{\footrulewidth}  {0.4pt}%

\vfill%
\begin{center}
  {\GRANDE Projeto EMMA} \\[5mm]
  {\Grande Robô para Inspeção de Turbinas In Situ.} \\[25mm]
  {\Grande Minutas de Reuniões} \\
  \vfill%
  {\Large \today} \\[8mm]
\end{center}

\newpage%
%---------------------------------------------------------------------
\pagestyle{fancy}%
\thispagestyle{fancy}%
\renewcommand{\headrulewidth}  {0.4pt}%
\renewcommand{\footrulewidth}  {0.4pt}%
%\lhead{\vspace*{-5mm}\includegraphics[bb=1 1 1023 392,width=30mm]{../logo/lead-logo.jpg}}%
\lhead{\vspace*{-5mm}\includegraphics[width=30mm]{figs/logo/lead-logo.pdf}}%
\chead{Projeto EMMA}%
\rhead{\sf\thepage}%
\lfoot{Minutas de Reuniões}%
\cfoot{}%
\rfoot{\sf [\hours] \quad \today}%
%---------------------------------------------------------------------
 
\tableofcontents

\newpage%
%---------------------------------------------------------------------
\section{Identificação}

%-----------------------------------------------------------------------
%
%   UFRJ  - Universidade Federal do Rio de Janeiro
%   COPPE - Coordena��o dos Programas de P�s-gradua��o em Engenharia
%   PEE   - Programa de Engenharia El�trica
%
%
%   Projeto ROSA - Rob� para opera��o de stoplogs alagados
%
%   Identifica��o
%                                                         Ramon R. Costa
%                                                         20/mar/14, Rio
%-----------------------------------------------------------------------
%\section{Identifica��o}
\dado{Título}{
  EMMA - Robô para Inspeções In Situ \\
}

\dado{Proponente}{
  Universidade Federal do Rio de Janeiro (UFRJ) \\[2mm]
  Fundação Coordenação de Projetos, Pesquisas e Estudos Tecnológicos (COPPETEC)
  \\
}

\dado{Contratante}{
  Energia Sustentável do Brasil S.A. \\
}

\dado{Execução}{
  Grupo de Simulação e Controle em Automação e Robótica (GSCAR) \\
}

\dado{Contrato}{
  Jirau 09/15 \\
}

\dado{P\&D ANEEL}{
  6631-0003/2015 \\
}

\dado{COPPETEC}{
  PEE 18.951 \\
}

\dado{Início}{
   02 de Março de 2015 \\
}

\dado{Prazo}{
  14 meses \\
}

\dado{Orçamento}{
   R\$ 2.487.473,47 \\
 }

\dado{Coordenador}{
  Ramon Romankevicius Costa \\
}

\dado{Gerente}{
  Breno Bellinati de Carvalho \\
}

%---------------------------------------------------------------------
\fim


\newpage%
%---------------------------------------------------------------------
\section{Equipe}


Alana Monteiro\\
Eduardo Elael\\
Estevão Fróes\\
Gabriel Alcantara\\
Julia Campana\\
Renan Freitas\\
Ramon Costa Romankevicius\\


\newpage%
%---------------------------------------------------------------------
\section{Minutas}
 
\subsection{Marco/2015}
  %---------------------------------------------------------------------
\subsubsection{Minuta de reunião (26-mar-2015)}

\begin{tabbing}
  Local \= xxx \kill
  Local \> : LEAD \\
  Data  \> : 26 de Março de 2015 \\
  Hora  \> : 10:00
\end{tabbing}

%---------------------------------------------------------------------
\participantes{
  \elael,
  \gabriel,
  \julia,
  \patrick,
  \alana,
  \ramon,
  \renan.
}

\begin{itemize}
  \item Aprovação da minuta.

  \item Discutir tarefas e recomendaçõees da equipe para essa semana.


  \item Update semanal. Início de atividades do Projeto EMMA.
  \begin{itemize}

    \item \textbf{\alana.} Tarefas: Entrevista Rafael e Jan Quinta-Feira as 13:00 pm 
    
    \item \textbf{\renan.} Tarefa: Adicionar parte de Elael ao User Manual.
    \item \textbf{\elael.} SOTA Article
    		\begin{itemize}    
			 \item Questoes comuns do problema
			 \item Consolidar possiveis solucoes 
    		\end{itemize}
    \item \textbf{\gabriel.} SOTA Article
    		\begin{itemize}    
			 \item Questoes comuns do problema
			 \item Consolidar possiveis solucoes 
			\end{itemize}
  \end{itemize}

%   \item Problemas em aberto:
%   \begin{itemize}
%     \item Procedimento de compras e medidas para as instalações finais do
%     laboratório.
%     \item Opções para a comprar de software, Adobe/ Solid Works/ Live
%     Meeting/Bibliografia
%     \item Fechar orçamento Inventário.
%     \item Criar log para documentar problemas de ROCK/ criar um forum para
%     colaboração (?!)
%     \item Dropbox para compartilhamento de arquivos.
%     \item Viagem ESB/Relatório:
%   \end{itemize}

  \item Agenda para a próxima reunião:
  \begin{itemize}
    \item Resultado de pesquisas individuais.
    \item Viagem Jirau Abril.
    \item Novas tarefas \& recomendações.
  \end{itemize}

\end{itemize}

\vspace{5mm}%
\parbox[t]{70mm}{
  Aprovado por: \\[5mm]
  \centering
  \includegraphics[width=65mm]{figs/logo/assinatura-ramon.png} \\[-4mm]
  \rule[2mm]{70mm}{0.1mm} \\
  \ramon \\[1mm]
  Coordenador do Projeto \\
}

%---------------------------------------------------------------------
\fim



 
\subsection{Abril/2015}
  \subsubsection{Minuta de reunião (16-Abril-2015)}

\begin{tabbing}
  Local \= xxx \kill
  Local \> : LEAD \\
  Data  \> : 16 de Abril de 2015 \\
  Hora  \> : 10:00
\end{tabbing}

%---------------------------------------------------------------------
\participantes{
  \elael,
  \alana,
  \gabriel,
  \julia,
  \ramon,
  \renan.
}

\begin{itemize}
  \item Aprovação da minuta.

  \item Update semanal do Projeto EMMA.
  
  \begin{itemize}
    \item \textbf{\renan.} 
		\begin{itemize}    
			 \item Reportcomp completo.
			 \item Relatório de Eletrônica
			 \item EMMA SOTA 
			
		\end{itemize}
		
		
    \item \textbf{\elael.} 
    		\begin{itemize}    
			 \item Lista de .xml
			 \item Testes executados.
			 \item Proposta Mestrado
			\end{itemize}
					
			
    \item \textbf{\gabriel.} 
    		\begin{itemize}    
			 \item Lista de .xml
			 \item Testes executados.
			 \item Proposta Mestrado
			 \item EMMA SOTA
			\end{itemize}

		 \item \textbf{\julia.} 
    		\begin{itemize}    
			 \item Desenhos de Conceito
			 \item Opções de Mestrado e aplicaçõespossíveis no projeto.
			\end{itemize}

			
  \end{itemize}


  \item Agenda para a próxima reunião:
  \begin{itemize}
    \item Resultado de pesquisas individuais.
    \item Viagem Jirau Abril.
    \item Novas tarefas \& recomendações.
  \end{itemize}

\end{itemize}

\vspace{5mm}%
\parbox[t]{70mm}{
  Aprovado por: \\[5mm]
  \centering
  \includegraphics[width=65mm]{figs/logo/assinatura-ramon.png} \\[-4mm]
  \rule[2mm]{70mm}{0.1mm} \\
  \ramon \\[1mm]
  Coordenador do Projeto \\
}

%---------------------------------------------------------------------
\fim
  
\subsection{Maio/2015} 
  \subsubsection{Minuta de reunião (07-Maio-2015)}

\begin{tabbing}
  Local \= xxx \kill
  Local \> : LEAD \\
  Data  \> : 07 de Maio de 2015 \\
  Hora  \> : 10:00
\end{tabbing}

%---------------------------------------------------------------------
\participantes{
  \elael,
  \alana,
  \gabriel,
  \julia,
  \ramon,
  \renan.
}

\textbf{Aprovação da minuta}

\textbf{Update semanal do Projeto EMMA}
  

 \textbf{\alana.} 
	\begin{itemize}
		\item \textbf{Tarefas concluídas:}
			\begin{itemize}    
				\item Diárias e administrativo de viagem.
			\end{itemize}
		
		\item \textbf{Novas tarefas:}
			\begin{itemize} 
				\item Dados ESBR
			\end{itemize}
	\end{itemize}
  
  
\textbf{\renan.} 
	\begin{itemize}
		\item \textbf{Tarefas concluídas:}
			\begin{itemize}    
				\item Analise técnica feita durante a viagem.
			\end{itemize}
		
		\item \textbf{Novas tarefas:}
			\begin{itemize} 
				\item Formalizar análise no EMMA SOTA
				\item conceito escoltilha inferior.
			\end{itemize}
	\end{itemize}
		
\textbf{\elael.} 
	\begin{itemize}
		\item \textbf{Tarefas concluídas:}
			\begin{itemize}    
				\item Ajustes de conceito da escotilha superior.
			\end{itemize}
		
		\item \textbf{Novas tarefas:}
			\begin{itemize} 
				\item Formalizar ajustes no EMMA SOTA
				\item Relatório de viagem Latex.
			\end{itemize}
	\end{itemize}
					
			
   \textbf{\gabriel.} 
	\begin{itemize}
		\item \textbf{Tarefas concluídas:}
			\begin{itemize}    
				\item Analise técnica feita durante a viagem.
			\end{itemize}
		
		\item \textbf{Novas tarefas:}
			\begin{itemize} 
			    \item Conceito Caixa.
				\item Formalizar ajustes no EMMA SOTA
			\end{itemize}
	\end{itemize}

			



\textbf{Agenda para a próxima reunião:}
  \begin{itemize}
    \item Resultado de pesquisas individuais.
    \item Novas tarefas \& recomendações.
  \end{itemize}


\vspace{5mm}%
\parbox[t]{70mm}{
  Aprovado por: \\[5mm]
  \centering
  \includegraphics[width=65mm]{figs/logo/assinatura-ramon.png} \\[-4mm]
  \rule[2mm]{70mm}{0.1mm} \\
  \ramon \\[1mm]
  Coordenador do Projeto \\
}

%---------------------------------------------------------------------
\fim
  \subsubsection{Minuta de reunião (26-mar-2015)}

\begin{tabbing}
  Local \= xxx \kill
  Local \> : LEAD \\
  Data  \> : 14 de Maio de 2015 \\
  Hora  \> : 10:00
\end{tabbing}

%---------------------------------------------------------------------
\participantes{
  \elael,
  \alana,
  \gabriel,
  \julia,
  \ramon,
  \renan.
}

\begin{itemize}
  \item Aprovação da minuta.

  \item Update semanal do Projeto EMMA.
  \begin{itemize}

  
    \item \textbf{\renan.} 
           \\Fez Apresentação sobre seu conceito Escotilha Inferior

\begin{itemize}    
			 \item Questões Gerais
			 \item Pros & Cons
			 \item Soluções de Logística 
			 \item Soluções de Robótica
			\end {itemize}
            \\
    \item \textbf{\elael.} 
    		\\SOTA Article
    		\begin{itemize}    
			 \item Questoes comuns do problema
			 \item Consolidar possiveis solucoes 
			\end {itemize}
			\\

    \item \textbf{\gabriel.} C\\SOTA Article
    		\begin{itemize}    
			 \item Questoes comuns do problema
			 \item Consolidar possiveis solucoes 
			\end {itemize}
			\\
  \end{itemize}

%   \item Problemas em aberto:
%   \begin{itemize}
%     \item Procedimento de compras e medidas para as instalações finais do
%     laboratório.
%     \item Opções para a comprar de software, Adobe/ Solid Works/ Live
%     Meeting/Bibliografia
%     \item Fechar orçamento Inventário.
%     \item Criar log para documentar problemas de ROCK/ criar um forum para
%     colaboração (?!)
%     \item Dropbox para compartilhamento de arquivos.
%     \item Viagem ESB/Relatório:
%   \end{itemize}

  \item Agenda para a próxima reunião:
  \begin{itemize}
    \item Resultado de pesquisas individuais.
    \item Viagem Jirau Abril.
    \item Novas tarefas \& recomendações.
  \end{itemize}

\end{itemize}

\vspace{5mm}%
\parbox[t]{70mm}{
  Aprovado por: \\[5mm]
  \centering
  \includegraphics[width=65mm]{../assinatura/assinatura-digital.jpg} \\[-4mm]
  \rule[2mm]{70mm}{0.1mm} \\
  \ramon \\[1mm]
  Coordenador do Projeto \\
}

%---------------------------------------------------------------------
\fim
  \subsubsection{Minuta de reunião (21-Maio-2015)}

\begin{tabbing}
  Local \= xxx \kill
  Local \> : LEAD \\
  Data  \> : 21 de Maio de 2015 \\
  Hora  \> : 10:00
\end{tabbing}

%---------------------------------------------------------------------
\participantes{
  \elael,
  \alana,
  \gabriel,
  \julia,
  \ramon,
  \renan.
}

  
\begin{itemize}
  \item Aprovação da minuta.

  \item Update semanal do Projeto EMMA.
  
  
    \item \textbf{\renan.} 
		\begin{itemize}    
			 \item Apresentação conceito Escotilha Inferior
			 \item Correções EMMA SOTA 
			
		\end{itemize}
		
		
    \item \textbf{\elael.} 
    		\begin{itemize}    
			 \item Apresentação conceito Escotilha Superior. 
			 \item Correções EMMA SOTA 
			\end{itemize}
					
			
    \item \textbf{\gabriel.} 
    		\begin{itemize}    
		 \item Apresentação conceito Caixa. 
		 \item Correções EMMA SOTA. 
			\end{itemize}

		 \item \textbf{\julia.} 
    		\begin{itemize}    
			 \item Relatórios ESBR
			 \item PR/Site/Midia Social Projeto
			 \item Livro de Atas
		
			\end{itemize}


  \item Agenda para a próxima reunião:
  \begin{itemize}
    \item Resultado de pesquisas individuais.
    \item Novas tarefas \& recomendações.
  \end{itemize}

\end{itemize}

\vspace{5mm}%
\parbox[t]{70mm}{
  Aprovado por: \\[5mm]
  \centering
  \includegraphics[width=65mm]{figs/logo/assinatura-ramon.png} \\[-4mm]
  \rule[2mm]{70mm}{0.1mm} \\
  \ramon \\[1mm]
  Coordenador do Projeto \\
}

%---------------------------------------------------------------------
\fim
  %---------------------------------------------------------------------
\subsubsection{Minuta de reunião (28-maio-2015)}

\begin{tabbing}
  Local \= xxx \kill
  Local \> : LEAD \\
  Data  \> : 28 de Maio de 2015 \\
  Hora  \> : 10:00
\end{tabbing}

%---------------------------------------------------------------------
\participantes{
  \elael,
  \gabriel,
  \julia,
  \patrick,
  \alana,
  \ramon,
  \renan.
}

\textbf{Aprovação da minuta}

\textbf{Update semanal do Projeto EMMA}
  
\textbf{\renan.} 
	\begin{itemize}
		\item \textbf{Tarefas concluídas:}
			\begin{itemize}    
				\item Ajustes de conceito da escotilha inferior.
			\end{itemize}
		
		\item \textbf{Novas tarefas:}
			\begin{itemize} 
				\item Formalizar ajustes no EMMA SOTA
			\end{itemize}
	\end{itemize}
	
\textbf{\elael.} 
    \begin{itemize}    
		\item \textbf{Tarefas concluídas:}
			\begin{itemize}
				\item Análise de torc e vibração.
				\item Explorou possibilidades para menosr vibração.
				\item Possibilidades relacionadas ao tamanho de braço do robô
			\end{itemize}
			
		\item \textbf{Novas tarefas:}
			\begin{itemize} 
				\item ver com Ramon se será necessário o uso de um 'demper'ou não.
				\item Ver a menos velocidade na qual a base permitirá a continuidade do
				processo de coating.
				\item Frmalizar alterações no SOTA.
			\end{itemize}
	\end{itemize}
	
\textbf{\gabriel.} 
	\begin{itemize}
		\item \textbf{Tarefas concluídas:}
			\begin{itemize}    
				\item Problemas relacionados ao ambiente da unidade geradora.
				\item Ë preciso entender como a curvatura do ambiente pode alterar a
				estabilidade do braço.
			\end{itemize}
		\item \textbf{Novas tarefas:}
			\begin{itemize}
				\item Parafusos nmmagnéticos: qual teria de ser o peso para segurar a base.
				Efeitos colaterais de agua, pressão e deformidade do ambiente.
				\item Frmalizar alterações no SOTA.		
			\end{itemize}
	\end{itemize}
	
\textbf{\julia.} 
	\begin{itemize}
		\item \textbf{Tarefas concluídas:}
			\begin{itemize}    
				\item Relatório Administrativo concluído.
				\item Entrevistas com orientadores de Mestrado. 
			\end{itemize}
		\item \textbf{Novas tarefas:}
			\begin{itemize}
				\item Apresentação EMMA, formalizar  projeto para divulgação.
				\item Roteiro para vídeo EMMA Aevo.
			\end{itemize}
	\end{itemize}
%   \item Problemas em aberto:
%   \begin{itemize}
%     \item Procedimento de compras e medidas para as instalações finais do
%     laboratório.
%     \item Opções para a comprar de software, Adobe/ Solid Works/ Live
%     Meeting/Bibliografia
%     \item Fechar orçamento Inventário.
%     \item Criar log para documentar problemas de ROCK/ criar um forum para
%     colaboração (?!)
%     \item Dropbox para compartilhamento de arquivos.
%     \item Viagem ESB/Relatório:
%   \end{itemize}

\textbf{Agenda para a próxima reunião:}
	\begin{itemize}
		\item Resultado de pesquisas individuais.
	    \item Viagem Jirau Abril.
	    \item Novas tarefas \& recomendações.
	\end{itemize}


\vspace{5mm}%
\parbox[t]{70mm}{
  Aprovado por: \\[5mm]
  \centering
  \includegraphics[width=65mm]{figs/logo/assinatura-ramon.png} \\[-4mm]
  \rule[2mm]{70mm}{0.1mm} \\
  \ramon \\[1mm]
  Coordenador do Projeto \\
}

%---------------------------------------------------------------------
\fim


 
% \newpage%
\subsection{Junho/2015}
   \subsubsection{Minuta de reunião (11-Junho-2015)}

\begin{tabbing}
  Local \= xxx \kill
  Local \> : LEAD \\
  Data  \> : 11 de Junho de 2015 \\
  Hora  \> : 9:00
\end{tabbing}

%---------------------------------------------------------------------
\participantes{
  \elael,
  \alana,
  \gabriel,
  \julia,
  \ramon,
  \renan.
}

\textbf{Aprovação da minuta}

\textbf{Update semanal do Projeto EMMA}
   
\textbf{\julia.} 
	\begin{itemize}
		\item \textbf{Tarefas concluídas:}
			\begin{itemize}    
				\item Apresentação EMMA.
				\item ADM/ Documentos /Viagem
			\end{itemize}
		
		\item \textbf{Novas tarefas:}
			\begin{itemize} 
				\item Atualizar documentação.
				\item Finalizar apresentação.
				\item Mestrado: questões relaciondas ao tema de controle de missão
				robótica.
			\end{itemize}
	\end{itemize}
					
		
\textbf{\elael.} 
	\begin{itemize}
		\item \textbf{Tarefas concluídas:}
			\begin{itemize}    
				\item Checou menor velocidade que possível para a base que permite a
				aplicação continua de coating.
				\item \item Calculou valores de torc.
			\end{itemize}
		
		\item \textbf{Novas tarefas:}
			\begin{itemize} 
				\item Formalizar ajustes no EMMA SOTA.
				\item Estudo sobre Shutter.
				\item Auxiliar estevão no desenho da Base.
			\end{itemize}
	\end{itemize}
					
\textbf{\gabriel.} 
	\begin{itemize}
		\item \textbf{Tarefas concluídas:}
			\begin{itemize}    
				\item analisouparafusos magnéticos.
				\item Modelo 3D em OpenRave.
			\end{itemize}
		
		\item \textbf{Novas tarefas:}
			\begin{itemize} 
				\item Formalizar ajustes no EMMA SOTA.
				\item Checar viabilidade de operaçào no espaço atrás das pás.
				\item Verificar manipuladores.
			\end{itemize}
	\end{itemize}
					
			
   \textbf{\estevão.} 
	\begin{itemize}
		\item \textbf{Tarefas concluídas:}
			\begin{itemize}    
				\item Trabalhou no modelo 3D da unidade geradora.
			\end{itemize}
		
		\item \textbf{Novas tarefas:}
			\begin{itemize} 
			    \item Terminar modelo 3D da unidade geradora.
				\item Desenho da base com Elael.
			\end{itemize}
	\end{itemize}

			



\textbf{Agenda para a próxima reunião:}
  \begin{itemize}
    \item Resultado de pesquisas individuais.
    \item Novas tarefas \& recomendações.
  \end{itemize}


\vspace{5mm}%
\parbox[t]{70mm}{
  Aprovado por: \\[5mm]
  \centering
  \includegraphics[width=65mm]{figs/logo/assinatura-ramon.png} \\[-4mm]
  \rule[2mm]{70mm}{0.1mm} \\
  \ramon \\[1mm]
  Coordenador do Projeto \\
}

%---------------------------------------------------------------------
\fim
   \subsubsection{Minuta de reunião (17-Junho-2015)}

\begin{tabbing}
  Local \= xxx \kill
  Local \> : LEAD \\
  Data  \> : 17 de Junho de 2015 \\
  Hora  \> : 13:00
\end{tabbing}

%---------------------------------------------------------------------
\participantes{
  \elael,
  \gabriel,
  \julia,
  \ramon.
}

\textbf{Aprovação da minuta}

\textbf{Update semanal do Projeto EMMA}
  
		
\textbf{\elael.} 
	\begin{itemize}
		\item \textbf{Tarefas concluídas:}
			\begin{itemize}    
				\item Estudo do Shutter: ainda espera info da Rijeza com valores
				relacionados a chama (trabalhou com estimativas árbitrárias que nos foram
				dadas anteriormente 3mm/ chama 3mil graus/ 230 mm).
				\item Identificou a classe de materiais (cerâmicas) de resistência para
				altas temperaturas.
				\item Abrir novas possibilidades para o design do shutter.
				\item Auxiliou estevão no elaboração do desenho da base.
			\end{itemize}
		
		\item \textbf{Novas tarefas:}
			\begin{itemize} 
				\item Formalizar ajustes no EMMA SOTA
				\item Atualizar estudo do 'Shutter'com numeros da Rijeza.
				\item Esboço do Shutter com Estevão.
				\item analizar shadow plates.
			\end{itemize}
	\end{itemize}
					
			
   \textbf{\gabriel.} 
	\begin{itemize}
		\item \textbf{Tarefas concluídas:}
			\begin{itemize}    
				\item workspace availability: KUKA 30L16 não é compatível com o ambiente
				(usou 45 graus como referência.) Maniulador  KUKA com 3m de alcance 30 L16,
				para frente da pá funciona bem, porém atras da pa fica incompatível com o
				espaço que temos na unidade geradora.
				\item KUKA KR30 bugado no openRave.
				\item Criou modelo simplificado com cilindros para facilitar KR30, mas ainda
				não acertou todos os eixos, simulação em processo de refinamento.
			\end{itemize}
		
		\item \textbf{Novas tarefas:}
			\begin{itemize} 
			    \item Manipuladores: Verificar modelo novo do motoman, 8 graus de
			    liberdade
			    \item Refinar simulação para incluir qualquer robô.
				\item Formalizar ajustes no EMMA SOTA.
			\end{itemize}
	\end{itemize}
	
	   \textbf{\julia.} 
	\begin{itemize}
		\item \textbf{Tarefas concluídas:}
			\begin{itemize}    
				\item Apresentação de projeto com conteúdo final para feedback de equipe, 
				Ramon e Patrick.
				\item Administrativo (seguro de vida, iniciação científica)
				\item Documentação de projeto atualizada.
			\end{itemize}
		
		\item \textbf{Novas tarefas:}
			\begin{itemize} 
			    \item Assimilar feedback de orientadora na proposta de mestrado
				\item Apresentação coordenada para viagem.
			\end{itemize}
	\end{itemize}

  \textbf{\estevão.} 
	\begin{itemize}
		\item \textbf{Tarefas concluídas:}
			\begin{itemize}    
				\item Elaborou conceito da base com Ramon, desenho em andamento.
				\item Concluiu desenho da unidade geradora.
			\end{itemize}
		
		\item \textbf{Novas tarefas:}
			\begin{itemize} 
			    \item Esboço do shutter.
			    \item Desenho da base.
			\end{itemize}
	\end{itemize}
			



\textbf{Agenda para a próxima reunião:}
  \begin{itemize}
    \item Solução é formada por: base totalmente retrátil + manipulador KUKA
    Lightweight LB820 + Shutter.
    \item Novas tarefas \& recomendações.
  \end{itemize}


\vspace{5mm}%
\parbox[t]{70mm}{
  Aprovado por: \\[5mm]
  \centering
  \includegraphics[width=65mm]{figs/logo/assinatura-ramon.png} \\[-4mm]
  \rule[2mm]{70mm}{0.1mm} \\
  \ramon \\[1mm]
  Coordenador do Projeto \\
}

%---------------------------------------------------------------------
\fim
   \subsubsection{Minuta de reunião (24-Junho-2015)}

\begin{tabbing}
  Local \= xxx \kill
  Local \> : LEAD \\
  Data  \> : 24 de Junho de 2015 \\
  Hora  \> : 13:00
\end{tabbing}

%---------------------------------------------------------------------
\participantes{
  \elael,
  \gabriel,
  \julia,
  \renan,
  \alana,
  \estevão,
  \ramon.
}

\textbf{Aprovação da minuta}

\textbf{Update semanal do Projeto EMMA}
  
		
\textbf{\elael.} 
	\begin{itemize}
		\item \textbf{Tarefas concluídas:}
			\begin{itemize}    
				\item Análise do Shutter em andamento com novos números Rijeza.
				\item Análise de material de alta resistência. 
			\end{itemize}
		
		\item \textbf{Novas tarefas:}
			\begin{itemize} 
				\item SOTA: finalizar conceito para deadline de quarta-feira 
				\item Cerâmicas de compressão: descobrir o material que a Rijeza usa.
				\item Email Darlan: Informação sobre a chama e pistola de coating.
			\end{itemize}
	\end{itemize}
	
	\textbf{\renan.} 
	\begin{itemize}
		\item \textbf{Novas tarefas:}
			\begin{itemize} 
				\item Se unir a Gabriel para finalizar a workspace analysis.
				\item Formalizar conclusões no SOTA para deadline.
			\end{itemize}
	\end{itemize}
					
			
   \textbf{\gabriel.} 
	\begin{itemize}
		\item \textbf{Tarefas concluídas:}
			\begin{itemize}    
				\item Bugs no workspace analysis resolvidos.
				\item Análise motoman em andamento.
			\end{itemize}
		
		\item \textbf{Novas tarefas:}
			\begin{itemize} 
			    \item Manipuladores: Verificar modelo novo do motoman, 8 graus de
			    liberdade.
			    \item Resolver 'colisions issues' Open Rave.
				\item Adicionar conclusões no SOTA.
			\end{itemize}
	\end{itemize}
	
	   \textbf{\julia.} 
	\begin{itemize}
		\item \textbf{Tarefas concluídas:}
			\begin{itemize}    
				\item Apresentação
				\item Logo do Projeto EMMA
				\item Documentação de projeto atualizada.
			\end{itemize}
		
		\item \textbf{Novas tarefas:}
			\begin{itemize} 
			    \item Revisão Proposta com Patrick.
			    \item Apresentação EMMA. 
			    \item Cotações viagens.
			\end{itemize}
	\end{itemize}

  \textbf{\estevão.} 
	\begin{itemize}
		\item \textbf{Tarefas concluídas:}
			\begin{itemize}    
				\item Slides com SolidWorks da base para apresentação adicionados.
			\end{itemize}
		
		\item \textbf{Novas tarefas:}
			\begin{itemize} 
			    \item Continuar esboço do shutter.
			    \item Formalizar trabalho de mecânica no SOTA.
			\end{itemize}
	\end{itemize}
			



\textbf{Agenda para a próxima reunião:}
  \begin{itemize}
    \item Novas tarefas \& recomendações.
  \end{itemize}


\vspace{5mm}%
\parbox[t]{70mm}{
  Aprovado por: \\[5mm]
  \centering
  \includegraphics[width=65mm]{figs/logo/assinatura-ramon.png} \\[-4mm]
  \rule[2mm]{70mm}{0.1mm} \\
  \ramon \\[1mm]
  Coordenador do Projeto \\
}

%---------------------------------------------------------------------
\fim
%   \include{minuta-de-reuniao(18-nov-2013)}
%   \include{minuta-de-reuniao(25-nov-2013)}

% \newpage%
\subsection{Julho/2015}
    \subsubsection{Minuta de reunião (01-Julho-2015)}

\begin{tabbing}
  Local \= xxx \kill
  Local \> : LEAD \\
  Data  \> : 01 de Julho de 2015 \\
  Hora  \> : 13:00
\end{tabbing}

%---------------------------------------------------------------------
\participantes{
  \elael,
  \alana,
  \gabriel,
  \julia,
  \ramon,
  \renan.
}

\textbf{Aprovação da minuta}

\textbf{Update semanal do Projeto EMMA}

\textbf{Considerações Gerais EMMA:}
  \begin{itemize}
    \item Contratações previstas: 1 engenheiro de software, 1 engenheiro de
    eletrônica e um aluno de mestrado de controle.
    \item Primeiro Objetivo do quadrimestre: definir a solução detalhada.
    Análise de riscos e benefícios.
    \item Segundo Objetivo do Quadrimentre: Determinar a relação de posição do
    braço robótico e do ambiente.
    \item Terceiro Objetivo do Quadrimentre: elaboração de tarefas do robô em
    preparo para arquitetura de interface.
  \end{itemize}
  
  
\textbf{\renan.} 
	\begin{itemize}
		\item \textbf{Tarefas concluídas:}
			\begin{itemize}    
				\item SOTA.
			\end{itemize}
		
		\item \textbf{Novas tarefas:}
			\begin{itemize} 
				\item Detalhes do acesso do robô na escotilha inferior.
				\item Apresentação.
			\end{itemize}
	\end{itemize}
		
\textbf{\elael.} 
	\begin{itemize}
		\item \textbf{Tarefas concluídas:}
			\begin{itemize}    
				\item SOTA
			\end{itemize}
		
		\item \textbf{Novas tarefas:}
			\begin{itemize} 
				\item Pesquisar sensores. Localização e Octomap para ajudar no processo de
				de calibração do braço Robótico.
				\item Mapeamento de tarefas do robô para interface de usuário.
			\end{itemize}
	\end{itemize}
					
			
   \textbf{\gabriel.} 
	\begin{itemize}
		\item \textbf{Tarefas concluídas:}
			\begin{itemize}    
				\item SOTA.
			\end{itemize}
		
		\item \textbf{Novas tarefas:}
			\begin{itemize} 
			    \item Procurar OpenSource para auxiliar no processo de calibração do
			    braço robótico.
				\item Mapeamento de tarefas do robô para interface de usuário.
			\end{itemize}
	\end{itemize}

			
		
   \textbf{\estevão.} 
	\begin{itemize}
		\item \textbf{Tarefas concluídas:}
			\begin{itemize}    
				\item SOTA.
			\end{itemize}
		
		\item \textbf{Novas tarefas:}
			\begin{itemize} 
			    \item Definir conceito da base para suporte de manipulador na entrada da
			    escotilha inferior.
			\end{itemize}
	\end{itemize}
	
		
   \textbf{\estevão.} 
	\begin{itemize}
		\item \textbf{Tarefas concluídas:}
			\begin{itemize}    
				\item Apresentação
			\end{itemize}
		
		\item \textbf{Novas tarefas:}
			\begin{itemize} 
			    \item Estudo das tarefas do robô e seu mapeamento para a construção da
			    interface de usuário.
			\end{itemize}
	\end{itemize}

			



\textbf{Agenda para a próxima reunião:}
  \begin{itemize}
    \item Resultado de pesquisas individuais.
    \item Novas tarefas \& recomendações.
  \end{itemize}


\vspace{5mm}%
\parbox[t]{70mm}{
  Aprovado por: \\[5mm]
  \centering
  \includegraphics[width=65mm]{figs/logo/assinatura-ramon.png} \\[-4mm]
  \rule[2mm]{70mm}{0.1mm} \\
  \ramon \\[1mm]
  Coordenador do Projeto \\
}

%---------------------------------------------------------------------
\fim
    \subsubsection{Minuta de reunião (08-Julho-2015)}

\begin{tabbing}
  Local \= xxx \kill
  Local \> : LEAD \\
  Data  \> : 08 de Julho de 2015 \\
  Hora  \> : 13:00
\end{tabbing}

%---------------------------------------------------------------------
\participantes{
  \elael,
  \alana,
  \gabriel,
  \julia,
  \ramon,
  \renan.
}

\textbf{Aprovação da minuta}

\textbf{Update semanal do Projeto EMMA}

  
\textbf{\renan.} 
	\begin{itemize}
		\item \textbf{Tarefas concluídas:}
			\begin{itemize}    
				\item Pesquisa de braços robóticos com boa relaçào entre peso e alcance.
				KUKA KR10 com 6 graus de liberdade e 10kg de payload.
				\item Apresentação para Rijeza em Porto Alegre.
			\end{itemize}
		
		\item \textbf{Novas tarefas:}
			\begin{itemize} 
				\item Metodologia de análise para braços robóticos.
			\end{itemize}
	\end{itemize}
		
\textbf{\elael.} 
	\begin{itemize}
		\item \textbf{Tarefas concluídas:}
			\begin{itemize}    
				\item Estado da Arte para calibração de braços robóticos.
				\item Encontrou uma solução para reparo que menciona localização por 3D. 
			\end{itemize}
		
		\item \textbf{Novas tarefas:}
			\begin{itemize} 
				\item Encontrar uma forma de reduzir o peso dos cabos.(email Darlan)
			\end{itemize}
	\end{itemize}
					
			
   \textbf{\gabriel.} 
	\begin{itemize}
		\item \textbf{Tarefas concluídas:}
			\begin{itemize}    
				\item Estudo sobre sensores e seus respectivos drivers, pontos fracos e
				fortes, como se enquadrariam em nossa solução.  
			\end{itemize}
		
		\item \textbf{Novas tarefas:}
			\begin{itemize} 
			    \item Pesquisar Sensores 1D
			    \item Auxiliar workspace analisys de Rena e Estevão com Open Rave. 
			\end{itemize}
	\end{itemize}

			
\textbf{\estevão.} 
	\begin{itemize}
		\item \textbf{Tarefas concluídas:}
			\begin{itemize}    
				\item Pesquisa de braços robóticos com boa relaçào entre peso e alcance.
				KUKA KR10 com 6 graus de liberdade e 10kg de payload.
				\item Trabalhou com a hipótese de 4 posições para cobrir a pá.
			\end{itemize}
		
		\item \textbf{Novas tarefas:}
			\begin{itemize} 
				\item Metodologia de análise para braços robóticos.
				\item Estudar qual o alcance e graus de liberdade pra cobrir a pá.
			\end{itemize}
	\end{itemize}
	
		
   \textbf{\julia.} 
	\begin{itemize}
		\item \textbf{Tarefas concluídas:}
			\begin{itemize}    
				\item Estudo das tarefas do robô e seu mapeamento para a construção da
			    interface de usuário.
			    \item Apresentou organograma das tarefas do Robô com descrição de
				atividades da interface do usuário.
			\end{itemize}
		
		\item \textbf{Novas tarefas:}
			\begin{itemize} 
			 \item Fluxograma, pesquisar para descrever em fluxograma completo os
			 processos relacionados a calibração, Reparo, Metalização e Jatemaneto
			 \item Distinguir e detalhar cada processo de tarefas. 
			\end{itemize}
	\end{itemize}

			



\textbf{Agenda para a próxima reunião:}
  \begin{itemize}
    \item Resultado de pesquisas individuais.
    \item Novas tarefas \& recomendações.
  \end{itemize}


\vspace{5mm}%
\parbox[t]{70mm}{
  Aprovado por: \\[5mm]
  \centering
  \includegraphics[width=65mm]{figs/logo/assinatura-ramon.png} \\[-4mm]
  \rule[2mm]{70mm}{0.1mm} \\
  \ramon \\[1mm]
  Coordenador do Projeto \\
}

%---------------------------------------------------------------------
\fim 
    \subsubsection{Minuta de reunião (15-Julho-2015)}

\begin{tabbing}
  Local \= xxx \kill
  Local \> : LEAD \\
  Data  \> : 15 de Julho de 2015 \\
  Hora  \> : 13:00
\end{tabbing}

%---------------------------------------------------------------------
\participantes{
  \elael,
  \alana,
  \gabriel,
  \julia,
  \ramon,
  \renan.
}

\textbf{Aprovação da minuta}

\textbf{Update semanal do Projeto EMMA}

  
\textbf{\renan.} 
	\begin{itemize}
		\item \textbf{Tarefas concluídas:}
			\begin{itemize}    
				\item Repasse do relatório da viagem a Rijeza.
			\end{itemize}
		
		\item \textbf{Novas tarefas:}
			\begin{itemize} 
				\item Dar continuidade a análise de pás com Estevão.
				\item Formalizar descobertas de Porto Alegre, e dar início a um novo
				Journal.
			\end{itemize}
	\end{itemize}
		
\textbf{\elael.} 
	\begin{itemize}
		\item \textbf{Tarefas concluídas:}
			\begin{itemize}    
				\item Pesquisa de Sensores.
			\end{itemize}
		
		\item \textbf{Novas tarefas:}
			\begin{itemize} 
				\item Formalizar Pesquisa de Sensores.
			\end{itemize}
	\end{itemize}
					
			
   \textbf{\gabriel.} 
	\begin{itemize}
		\item \textbf{Tarefas concluídas:}
			\begin{itemize}    
				\item Formalizou pesquisa de Sensores de alta precisão com pros e cons:
				Pros: sensores não são 3d e sim fechos de laser com um espelho giratório e
				um PanTilt que varre o ambiente.
				Cons: Tais sensores demoram muito para fazer a varredura completa.
			\end{itemize}
		
	\end{itemize}

			
\textbf{\estevão.} 
	\begin{itemize}
		\item \textbf{Tarefas concluídas:}
			\begin{itemize}    
				\item Metodologia criada: malha de pontos, rotina que identifica o
				manipulador específico que atende aos parâmetros do projeto. Os
				manipuladores pesquisados estão sendo adicionados de acordo.
				\item Análise cinemática e dinâmica com OpenRave e MathLab para simulação
				dinâmica.
			\end{itemize}
		
		\item \textbf{Novas tarefas:}
			\begin{itemize} 
				\item Produzor gráfico para pá robótica.
				\item Maquete de braço.
				\item Opções de estagiário.
			\end{itemize}
	\end{itemize}
	
		
   \textbf{\julia.} 
	\begin{itemize}
		\item \textbf{Tarefas concluídas:}
			\begin{itemize}    
				\item Fluxograma com tarefas do robô descritas
			    \item Estudo de artigos para pesquisa de mapping e visualização de data.
			\end{itemize}
		
		\item \textbf{Novas tarefas:}
			\begin{itemize} 
			 \item Fluxograma Macro
			 \item Formalizar pesquisa de Mapping.
			 \item Entregar proposta revisada
			\end{itemize}
	\end{itemize}

			



\textbf{Agenda para a próxima reunião:}
  \begin{itemize}
    \item Resultado de pesquisas individuais.
    \item Novas tarefas \& recomendações.
  \end{itemize}


\vspace{5mm}%
\parbox[t]{70mm}{
  Aprovado por: \\[5mm]
  \centering
  \includegraphics[width=65mm]{figs/logo/assinatura-ramon.png} \\[-4mm]
  \rule[2mm]{70mm}{0.1mm} \\
  \ramon \\[1mm]
  Coordenador do Projeto \\
}

%---------------------------------------------------------------------
\fim
% 
% \newpage%
% \subsection{Janeiro/2014}
%   \input{minuta-de-reuniao(06-jan-2014)}
%   \include{minuta-de-reuniao(13-jan-2014)}
%   \include{minuta-de-reuniao(21-jan-2014)}
%   \include{minuta-de-reuniao(27-jan-2014)}
% 
% \newpage%
% \subsection{Fevereiro/2014}
%   \input{minuta-de-reuniao(03-fev-2014)}
%   \include{minuta-de-reuniao(18-fev-2014)} %Confirmar a data
%   \include{minuta-de-reuniao(14-fev-2014)}
%   \include{minuta-de-reuniao(24-fev-2014)}
% 
% \newpage%
% \subsection{Marco/2014}
%   \input{minuta-de-reuniao(11-mar-2014)} %Conferir! 11 ou 12/mar ???  Revisar com J�lia...

%---------------------------------------------------------------------

%---------------------------------------------------------------------
\end{document}

